%!TEX root = ../OGUSAdoc.tex

Three markets must clear in \ogindia---the labor market, the capital market, and the goods market. By Walras' Law, we only need to use two of those market clearing conditions because the third one is redundant. In the model, we choose to use the labor market clearing condition and the capital market clearing condition, and to ignore the goods market clearing condition. But we present all three market clearing conditions here. Further, the redundant goods market clearing condition---sometimes referred to as the resource constraint---makes for a nice check on the solution method to see if everything worked.

We also characterize here the law of motion for total bequests $BQ_t$. Although it is not technically a market clearing condition, one could think of the bequests law of motion as the bequests market clearing condition.

\section{Market Clearing Conditions}\label{SecMarkClrMktClr}

  Labor market clearing \eqref{EqMarkClrLab} requires that aggregate labor demand $L_t$ measured in efficiency units equal the sum of household efficiency labor supplied $e_{j,s}n_{j,s,t}$.
  \begin{equation}\label{EqMarkClrLab}
    L_t = \sum_{s=E+1}^{E+S}\sum_{j=1}^{J} \omega_{s,t}\lambda_j e_{j,s}n_{j,s,t} \quad \forall t
  \end{equation}
  Capital market clearing \eqref{EqMarkClrCap} requires that aggregate capital demand from firms $K_t$ and from the government $D_t$ equal the sum of capital savings and investment by households $b_{j,s,t}$.
  \begin{equation}\label{EqMarkClrCap}
    K_t + D_t = \sum_{s=E+2}^{E+S+1}\sum_{j=1}^{J}\Bigl(\omega_{s-1,t-1}\lambda_j b_{j,s,t} + i_s\omega_{s,t-1}\lambda_j b_{j,s,t}\Bigr) \quad \forall t
  \end{equation}
  Note that the capital demand side of the capital market clearing equation \eqref{EqMarkClrCap} includes both capital demand by firms $K_t$ and capital demand by government $D_t$. It is here that we can see the potential of government deficits to crowd out investment.

  Aggregate consumption $C_t$ is defined as the sum of all household consumptions, and aggregate investment is defined by the resource constraint $Y_t = C_t + I_t + G_t$ as shown in \eqref{EqMarkClrGoods}.
  \begin{equation}\label{EqMarkClrGoods}
    \begin{split}
      Y_t &= C_t + K_{t+1} - \biggl(\sum_{s=E+2}^{E+S+1}\sum_{j=1}^{J}i_s\omega_{s,t}\lambda_j b_{j,s,t+1}\biggr) - (1-\delta)K_t + G_t \quad\forall t \\
      &\quad\text{where}\quad C_t \equiv \sum_{s=E+1}^{E+S}\sum_{j=1}^{J}\omega_{s,t}\lambda_j c_{j,s,t}
    \end{split}
  \end{equation}

  Note that the extra terms with the immigration rate $i_s$ in the capital market clearing equation \eqref{EqMarkClrCap} and the resource constraint \eqref{EqMarkClrGoods} accounts for the assumption that age-$s$ immigrants in period $t$ bring with them (or take with them in the case of out-migration) the same amount of capital as their domestic counterparts of the same age. Note also that the term in parentheses with immigration rates $i_s$ in the sum acts is equivalent to a net exports term in the standard equation $Y=C+I+G+NX$. That is, if immigration rates are positive, then immigrants are bringing capital into the country and the term in parentheses has a negative sign in front of it. Negative exports are imports.


\section{Total Bequests Law of Motion}\label{SecMarkClrBQ}

  Total bequests $BQ_t$ are the collection of savings of household from the previous period who died at the end of the period. These savings are augmented by the interest rate because they are returned after being invested in the production process.
  \begin{equation}\label{EqMarkClrBQ}
    BQ_{t} = (1+r_{t})\left(\sum_{s=E+2}^{E+S+1}\sum_{j=1}^J\rho_{s-1}\lambda_j\omega_{s-1,t-1}b_{j,s,t}\right) \quad\forall t
  \end{equation}
  Because the form of the period utility function in \eqref{EqHHPerUtil} ensures that $b_{j,s,t}>0$ for all $j$, $s$, and $t$, total bequests will always be positive $BQ_{j,t}>0$ for all $j$ and $t$.
