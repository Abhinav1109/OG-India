%!TEX root = ../OGUSAdoc.tex

The production side of the \ogindia model is populated by a unit measure of identical perfectly competitive firms that rent capital $K_t$ and hire labor $L_t$ to produce output $Y_t$. Firms also face a flat corporate income tax $\tau^{corp}$ as well as a tax on the amount of capital they depreciate $\tau^\delta$.


\section{Production Function}\label{EqFirmsProdFunc}

  Firms produce output $Y_t$ using inputs of capital $K_t$ and labor $L_t$ according to a general constant elasticity (CES) of substitution production function,
  \begin{equation}\label{EqFirmsCESprodfun}
    Y_t = F(K_t, L_t) \equiv Z_t\biggl[(\gamma)^\frac{1}{\ve}(K_t)^\frac{\ve-1}{\ve} + (1-\gamma)^\frac{1}{\ve}(e^{g_y t}L_t)^\frac{\ve-1}{\ve}\biggr]^\frac{\ve}{\ve-1} \quad\forall t
  \end{equation}
  where $Z_t$ is an exogenous scale parameter (total factor productivity) that can be time dependent, $\gamma$ represents the capital share of income, and $\ve$ is the constant elasticity of substitution between capital and labor. We have included constant productivity growth $g_y$ as the rate of labor augmenting technological progress.

  A nice feature of the CES production function is that the Cobb-Douglas production function is a nested case for $\ve=1$.
  \begin{equation}\label{EqFirmsCDprodfun}
    Y_t = Z_t(K_t)^\gamma(e^{g_y t}L_t)^{1-\gamma} \quad\text{for}\quad \ve=1 \quad\forall t
  \end{equation}


\section{Optimality Conditions}\label{EqFirmsFOC}

  The profit function of the representative firm is the following.
  \begin{equation}\label{EqFirmsProfit}
    PR_t = (1 - \tau^{corp})\Bigl[F(K_t,L_t) - w_t L_t\Bigr] - \bigl(r_t + \delta\bigr)K_t + \tau^{corp}\delta^\tau K_t \quad\forall t
  \end{equation}
  Gross income for the firms is given by the production function $F(K,L)$ because we have normalized the price of the consumption good to 1. Labor costs to the firm are $w_t L_t$, and capital costs are $(r_t +\delta)K_t$. The per-period economic depreciation rate is given by $\delta$.

  Taxes enter the firm's profit function \eqref{EqFirmsProfit} in two places. The first is the corporate income tax rate $\tau^{corp}$, which is a flat tax on corporate income. As is the case in the U.S., corporate income is defined as gross income minus labor costs. This will cause the corporate tax to only distort the firms' capital demand decision.

  The next place where tax policy enters the profit function \eqref{EqFirmsProfit} is through a refund of a percent of depreciation costs $\delta^\tau$ refunded at the corporate income tax rate $\tau^{corp}$. When $\delta^\tau=0$, no depreciation expense is deducted from the firm's tax liability. When $\delta^\tau=\delta$, all economic depreciation is deducted from corporate income.

  Taking the derivative of the profit function \eqref{EqFirmsProfit} with respect to labor $L_t$ and setting it equal to zero and taking the derivative of the profit function with respect to capital $K_t$ and setting it equal to zero, respectively, characterizes the optimal labor and capital demands.
  \begin{align}
    w_t &= e^{g_y t}(Z_t)^\frac{\ve-1}{\ve}\left[(1-\gamma)\frac{Y_t}{e^{g_y t}L_t}\right]^\frac{1}{\ve} \quad\forall t \label{EqFirmFOC_L} \\
    r_t &= (1 - \tau^{corp})(Z_t)^\frac{\ve-1}{\ve}\left[\gamma\frac{Y_t}{K_t}\right]^\frac{1}{\ve} - \delta + \tau^{corp}\delta^\tau \quad\forall t \label{EqFirmFOC_K}
  \end{align}

  We discuss how to calibrate the values of $\tau^{corp}$ and $\tau^\delta$ from the \btax microsimulation model in Chapter \ref{Chap_BTax}.
